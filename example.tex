% Options for packages loaded elsewhere
\PassOptionsToPackage{unicode}{hyperref}
\PassOptionsToPackage{hyphens}{url}
%
\documentclass[
  ignorenonframetext,
]{beamer}
\usepackage{pgfpages}
\setbeamertemplate{caption}[numbered]
\setbeamertemplate{caption label separator}{: }
\setbeamercolor{caption name}{fg=normal text.fg}
\beamertemplatenavigationsymbolsempty
% Prevent slide breaks in the middle of a paragraph
\widowpenalties 1 10000
\raggedbottom
\setbeamertemplate{part page}{
  \centering
  \begin{beamercolorbox}[sep=16pt,center]{part title}
    \usebeamerfont{part title}\insertpart\par
  \end{beamercolorbox}
}
\setbeamertemplate{section page}{
  \centering
  \begin{beamercolorbox}[sep=12pt,center]{part title}
    \usebeamerfont{section title}\insertsection\par
  \end{beamercolorbox}
}
\setbeamertemplate{subsection page}{
  \centering
  \begin{beamercolorbox}[sep=8pt,center]{part title}
    \usebeamerfont{subsection title}\insertsubsection\par
  \end{beamercolorbox}
}
\AtBeginPart{
  \frame{\partpage}
}
\AtBeginSection{
  \ifbibliography
  \else
    \frame{\sectionpage}
  \fi
}
\AtBeginSubsection{
  \frame{\subsectionpage}
}
\usepackage{lmodern}
\usepackage{amssymb,amsmath}
\usepackage{ifxetex,ifluatex}
\ifnum 0\ifxetex 1\fi\ifluatex 1\fi=0 % if pdftex
  \usepackage[T1]{fontenc}
  \usepackage[utf8]{inputenc}
  \usepackage{textcomp} % provide euro and other symbols
\else % if luatex or xetex
  \usepackage{unicode-math}
  \defaultfontfeatures{Scale=MatchLowercase}
  \defaultfontfeatures[\rmfamily]{Ligatures=TeX,Scale=1}
\fi
% Use upquote if available, for straight quotes in verbatim environments
\IfFileExists{upquote.sty}{\usepackage{upquote}}{}
\IfFileExists{microtype.sty}{% use microtype if available
  \usepackage[]{microtype}
  \UseMicrotypeSet[protrusion]{basicmath} % disable protrusion for tt fonts
}{}
\makeatletter
\@ifundefined{KOMAClassName}{% if non-KOMA class
  \IfFileExists{parskip.sty}{%
    \usepackage{parskip}
  }{% else
    \setlength{\parindent}{0pt}
    \setlength{\parskip}{6pt plus 2pt minus 1pt}}
}{% if KOMA class
  \KOMAoptions{parskip=half}}
\makeatother
\usepackage{xcolor}
\IfFileExists{xurl.sty}{\usepackage{xurl}}{} % add URL line breaks if available
\IfFileExists{bookmark.sty}{\usepackage{bookmark}}{\usepackage{hyperref}}
\hypersetup{
  pdftitle={Example Slides},
  hidelinks,
  pdfcreator={LaTeX via pandoc}}
\urlstyle{same} % disable monospaced font for URLs
\newif\ifbibliography
\setlength{\emergencystretch}{3em} % prevent overfull lines
\providecommand{\tightlist}{%
  \setlength{\itemsep}{0pt}\setlength{\parskip}{0pt}}
\setcounter{secnumdepth}{-\maxdimen} % remove section numbering

\title{Example Slides}
\author{}
\date{\vspace{-2.5em}}

\begin{document}
\frame{\titlepage}

\begin{frame}{Guidelines}
\protect\hypertarget{guidelines}{}

This presentation is an example on how you should submitt your slides.
You can find out more about our guidelines on our website
(\url{https://user2020muc.r-project.org/}).\\
The website also has a full list of all the guidelines, this
presentation only features a selection (?).

\begin{block}{R Markdown}

This is an R Markdown presentation. Markdown is a simple formatting
syntax for authoring HTML, PDF, and MS Word documents. For more details
on using R Markdown see \url{http://rmarkdown.rstudio.com}.\\
There also is a cheatsheet, see
\url{https://rstudio.com/wp-content/uploads/2016/03/rmarkdown-cheatsheet-2.0.pdf}.

Please submit your presentation/poster in markdown as it will enable
blind people to use screen reading software or braille displays to
follow the details (e.g.~code) of talks.

\end{block}

\end{frame}

\begin{frame}

\begin{block}{Requirements}

\begin{itemize}
\tightlist
\item
  Length:

  \begin{itemize}
  \tightlist
  \item
    contributed: 10-15 minutes
  \item
    lightning: 3-5 minutes
  \item
    poster: 0.5-1 minute
  \end{itemize}
\item
  Video should be picture in picture (slides in the background, speaker
  smaller in the foreground), or slides and speaker should be on
  splitted screens.
\item
  Please let us know if your video requires a content warning (e.g.~due
  to unavoidable presentation of offensive or sexualised material, or
  due to flashing images).
\end{itemize}

\end{block}

\end{frame}

\begin{frame}

\begin{block}{Video of the speaker}

\begin{itemize}
\tightlist
\item
  Best would be a neutral background and low or no background audio.
\item
  Your face should be clearly visible (people who are deaf or hard of
  hearing may be able to lip-read).
\item
  Speak clearly.
\item
  Please pace yourself, so the audience can integrate both audio and
  visual information. Graphics, pictures, videos, and memes should be
  described audibly.
\item
  Speak every word on a slide, read long excerpts aloud.
\item
  Verbally describe images.
\end{itemize}

\end{block}

\end{frame}

\begin{frame}

\begin{block}{Slides}

\begin{itemize}
\tightlist
\item
  Use large sans serif fonts (as a guide 28-32pt or above for regular
  text).
\item
  Use high contrast.
\item
  Make sure slides are discernable for color blind users.
\item
  Use more than color to communicate information (color coding cannot be
  understood by people who are blind or colorblind).
\item
  Do not use flashing videos or images.
\item
  Avoid using animations (Unless with a detailed audio description).
\item
  Provide a text equivalent for graphics, but not for graphics that are
  only meant for decoration.
\end{itemize}

\end{block}

\end{frame}

\end{document}
